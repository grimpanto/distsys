
\chapter{Entfernter Zugriff}


\section{FTP}

Datenzugriff mittels ftp (RFC 959) ist ein relativ altes und noch
immer sehr weit verbreitetes Protokoll. 

Anders als http ben�tzt ftp \emph{mehr} als eine Verbindung zwischen
Client und Server: Zun�chst baut der Client eine Verbindung zum Port
21 (Control Port) des Servers auf. Diese Verbindung dient zur Authentifizierung
und Befehls�bertragung. Zur eigentlichen Daten�bertragung wird bei
Bedarf eine eigene Verbindung aufgebaut.

Dazu gibt es \emph{2 Betriebsarten}:
\begin{description}
\item [{Active~Mode}] Der Server baut von seinem Port 20 (Data Port) eine
Verbindung zu einem vom Client gew�hlten Port (�ber 1023) auf. 
\item [{Passive~Mode}] Der Client baut eine Verbindung zu einem vom Server
gew�hlten Port auf (meistens beide gr��er 1023). 
\end{description}
Dieses Protokoll wird als relativ unsicher angesehen, da sowohl die
Passwort�bertragung als auch die Daten�bertragung unverschl�sselt
stattfindet. Technisch gesehen ist der Befehlskanal wie beim telnet
Protokoll aufgebaut. Aus diesem Grund sollte man Daten nur mit scp
(secure copy protocol) oder der Weiterentwicklung sftp (siehe \vref{sec:SSH})
arbeiten.


\section{Netzwerkdateisysteme}

Als Alternative zu einer �bertragung mit ftp, scp oder sftp kann f�r
den Datenzugriff ein Netzwerkdateisystem wie NFS (Network File System)
oder CIFS (Common Internet File System) verwendet werden.


\minisec{NFS}

NFS wurde von SUN entwickelt und stellt ein Netzwerkdateisystem f�r
Unix dar. NFS hat heutzutage keine dominante Bedeutung mehr. Viele
wichtige Punkte wie die Authentifizierung der Benutzer (und nicht
nur der Rechner wie in NFSv3) oder der Wegfall der Unix-Lastigkeit
wurden erst in NFSv4 eingef�hrt. Diese Version hat sich jedoch nicht
durchsetzen k�nnen.


\minisec{CIFS}

CIFS wurde 1996 von Microsoft bei der IETF eingereicht und stellt
eine Erweiterung von SMB (Server Message Blocks) dar. SMB wurde von
IBM mit dem Ziel entwickelt f�r DOS ein Netzwerkdateisystem zur Verf�gung
zu haben. Microsoft hat SMB weiterentwickelt und in ihre Produktlinie
integriert. CIFS bietet entfernten Zugriff auf Dateien, Drucker oder
andere Ger�ten als auch die M�glichkeit zur IPC.


\section{Entfernte Ausf�hrung}


\subsection{SSH\label{sec:SSH}}

siehe Foliensatz!

%% SSH ist nicht nur ein Protokoll (siehe Abschnitt \vref{sec:SSH_protocol}),
%% sondern auch eine Anwendung, die es erlaubt �ber das Netz sich auf
%% einem Rechner anzumelden und Befehle abzusetzen.

%% Es ersetzt unsichere Protokolle wie rsh, rlogin oder telnet. rcp kann
%% durch scp (secure copy) ersetzt werden und ftp durch sftp (secure
%% file transfer protocol). scp und sftp ersetzen die unverschl�sselten
%% Protokolle, haben ansonst aber keine Gemeinsamkeiten. Z.B. gibt es
%% bei sftp auch keinen getrennten Daten- und Befehlskanal. Es wird wie
%% f�r ssh der Port 22 verwendet.


\subsection{X-Window-Protokoll}

Mit dem X-Window-Protokoll kann man graphische Ausgaben zum Client
(dieser wird als Server bezeichnet) und Eingaben zum Server (das ist
der eigentliche Client) schicken.

�hnlich: Windows Terminal Server


