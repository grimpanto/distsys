
\chapter{Prozesse}


\section{Prozesse vs. Threads}


\minisec{Prozesse}
\begin{itemize}
\item Was ist ein \emph{Prozess}? 

\begin{itemize}
\item Programm besteht aus Anweisungen 
\item Prozess entsteht bei Ausf�hrung eines Programmes 
\item Ein Prozess hat einen oder mehrere Threads 
\item Aus allgemeiner Sicht ist jeder Prozess ein eigener Ablaufpfad (und
kann daher im Computer entweder durch einen Prozess oder durch einen
Thread repr�sentiert werden).


Wir verwenden im weiteren Text den Begriff Prozess sowohl f�r die
Bedeutung in allgemeiner Sicht als auch f�r die konkrete Sicht als
Betriebssystemprozess. 

\end{itemize}
\item \textbf{Vorteile}: Nichtbeinflussung durch andere Prozesse. Dies wird
erreicht, indem jedem Prozess ein eigener virtueller Prozessor mit
einem eigenen Adressraum vorgespielt wird. Abgesehen von dem Adressraum
greift der Prozessor nat�rlich auch auf die anderen Ressourcen im
System zu, die von anderen Prozessen genutzt werden. Dazu ben�tigt
das Betriebssystem Unterst�tzung seitens der Hardware (z.B. MMU). 
\item \textbf{Nachteile}: Diese Art der Nebenl�ufigkeitstransparenz hat
einen relativ hohen Preis: Bei jeder Erzeugung eines Prozesses muss
ein eigener unabh�ngiger Adressraum gebildet werden (Speichersegmente
initialisieren,...), ein Stack angelegt werden,...


Weiters ist auch die Umschaltung zwischen den Prozessen relativ aufw�ndig
(Context switching overhead): CPU Kontext (Register, Programmz�hler,
Stackpointer,...), MMU Register, Prozesse auf Festplatte auslagern
bzw. in den Hauptspeicher laden, CPU Abrechnungen erstellen,... 

\item Zur Kommunikation zwischen Prozessen (Interprocess Communication,
IPC) k�nnen Pipes, Shared Memory, Sockets (TCP) oder Datagrams (UDP)
oder auch Dateien verwendet werden.
\end{itemize}

\minisec{Threads}
\begin{itemize}
\item Was ist ein \emph{Thread}? 

\begin{itemize}
\item Jeder Thread bedeutet eigenen Ausf�hrungszweig \emph{innerhalb} eines
Prozesses. 
\item Basis f�r effiziente und robuste Server 
\item zwischen Threads kann direkt auf Variablen zugegriffen werden (bei
Prozessen nicht). 
\item Zugriff auf die lokalen Daten eines anderen Thread nicht m�glich (Stack,
primitive vs. Objekttypen!) 
\end{itemize}
\item \textbf{Vorteile}: geringerer Overhead (es m�ssen haupts�chlich die
CPU Register gewechselt werden), Zugriff auf Daten und offene Dateien,... 
\item \textbf{Nachteile}: Beeinflussung durch andere Threads 
\item Kommunikation zwischen Threads: wie bei Prozessen und zus�tzlich �ber
gemeinsame Daten.
\end{itemize}
Prozesse und Threads vereinfachen einerseits die Programmierung, da
quasi-parallele (nur ein Prozessor, Zeitscheiben) oder auch parallele
Aktionen (mehrere Prozessoren) leicht programmiert werden k�nnen.
Als gemeinsamer Nachteil sind die Probleme zu nennen, die durch die
gemeinsame Kommunikation entstehen ($\rightarrow$ Synchronisation).

Beispiele f�r nebenl�ufige Anwendungen, die Threads oder Prozesse
verwenden (Client und/oder Server) sind Web services, Anwendungen
der nummerische Mathematik und Simulation, I/O Verarbeitung, GUI-basierte
Anwendungen (X-Window: Server/Client) und Embedded systems (Verwendung
in Echtzeitsystemen).


\section{Threads}

Verwendung finden Threads auch oft in nicht verteilten Systemen: haupts�chlich,
um bei dem Aufruf von \emph{blockierenden Systemaufrufen} nicht die
Gesamtanwendung zum Stillstand zu bringen.

Beispiele: 
\begin{itemize}
\item Zugriff auf Ein-/Ausgabeger�te, wie serielle Schnittstelle, das Netzwerk,
den Drucker, aber auch die Festplatte. 
\item Durchf�hren von langen Berechnungen bei GUI Anwendungen (z.B. einer
Tabellenkalkulation), damit weitere Benutzereingaben get�tigt werden
k�nnen. In so einem Fall werden meistens zwei Threads verwendet: einer
f�r die Interaktion und einer f�r die Berechnungen. 
\item Nutzung von mehreren Prozessoren in Multiprozessor-Maschinen. 
\end{itemize}
Weiters kann der Entwurf einfacher werden, da die Gesamtanwendung
in mehrere Teile zerlegt werden kann, wobei jeder Thread f�r einen
dedizierten Teil der Gesamtanwendung zust�ndig ist.

Obwohl die Erzeugung und Verwaltung von Threads weniger kostspielig
ist, als die entsprechenden Operationen f�r Prozesse, kann es durchaus
sinnvoll sein, auch hier Optimierungen vorzunehmen: Das Ziel ist es,
Threads nur so oft zu erzeugen wie dies unbedingt notwendig ist --
dies wird mit Threadpooling erreicht.

 
