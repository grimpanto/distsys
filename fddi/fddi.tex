
\chapter{FDDI}

FDDI (Fiber Distributed Data Interface) weist gro�e �hnlichkeiten
mit dem Tokenring auf. Die Datenrate ist 100MBit/s. Es gibt auch eine
Variante mit einer Datenrate von 1000MBit/s. Ringl�ngen sind bis zu
100 oder 200km m�glich.


\section{Prinzip}

In Abweichung zum Tokenring werden optische Fasern in einem Doppelring
verwendet. Bei FDDI kreist ebenfalls ein Token im Kreis, im Unterschied
zum Tokenring k�nnen jedoch auch mehrere Tokens gleichzeitig kreisen.
Dazu gibt der Sender das Token sofort nach dem Senden des letzten
Frames wieder frei. Das Entfernen der Rahmen wird ebenfalls nur vom
Absender durchgef�hrt.


\section{Netzaufbau}

Der Doppelring besteht aus zwei Ringen, die gegel�ufig betrieben werden
und als Haupt- und Ersatzring ausgelegt sind. Bei einer Unterbrechung
beider Ringe (an der selben Stelle) k�nnen die benachbarten Stationen
den Ring wieder schlie�en. Damit wirkt sich die Unterbrechung, abgesehen
von einer doppelt langen Ringl�nge nicht aus.

�hnlich den MSAUs im Tokenring definiert der FDDI Standard ebenfalls
Anschlussm�glichkeiten f�r die einzelnen Stationen. Allerdings werden
verschiedene Arten von Stationen definiert:
\begin{itemize}
\item Eine DAS (dual attachment station) verf�gt �ber zwei getrennte Anschl�sse,
sodass die Station unabh�ngig �ber beide Ringe kommunizieren kann.
Ebenfalls wie bei Tokenring wird bei Ausfall der Station der Ring
nicht unterbrochen.
\item Ein DAC (dual attachment concentrator) wird wie eine DAS an den Ring
angeschlossen. Das Besondere ist, dass an einen DAC mehrere DAS bzw.
SAS (siehe n�chsten Punkt) angeschlossen werden k�nnen.
\item Eine SAS (single attachment station) wird immer �ber eine DAS oder
DAC an den Ring angeschlossen.
\end{itemize}
DAC bzw. DAS sind direkt im Doppelring einh�ngbar. Eine SAS kann nicht
direkt in den Doppelring eingeh�ngt werden. Eine DAS kann auch an
zwei verschiedene DAC angeschlossen werden. Diese als Dual Homing
bezeichnete Konfiguration und erh�ht nochmals die Verf�gbarkeit der
Station am Netz.
