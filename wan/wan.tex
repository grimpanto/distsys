
\chapter{WAN Technologien}

In diesem Kapitel werden die wichtigsten WAN Technologien kurz vorgestellt.


\section{X.25}

Bei X.25 handelt es sich um ein �ffentliches, paketvermittelndes Netzwerk,
das im Jahr 1976 standardisiert wurde. Eigentlich beschreibt X.25
die Schnittstelle zwischen Endger�t und Netz. X.25 Netze stellen permanente
virtuelle Verbindungen oder geschaltete virtuelle Verbindungen zur
Verf�gung. An Datenraten sind 300Bit/s bis 64KBit/s definiert und
seit 1992 gibt es auch 2MBit/s. Die Datenraten entsprechen nicht mehr
den heutigen Anforderungen an ein WAN. Der Grund f�r die niedrigen
Datenraten liegt einerseits darin, dass die Technologie relativ alt
ist und X.25 f�r die Verwendung �ber schlechte analoge �bertragungsstrecken
ausgelegt wurde. Der gro�e Vorteil dieser Technologie ist, dass sie
weltweit verf�gbar ist.


\section{ISDN}

ISDN steht f�r Integrated Services Digital Network und hat das Ziel,
verschiedene Dienste wie Sprach-, Daten- und Bildkommunikation in
einem Netz zur Verf�gung zu stellen. Es wurde im Jahr 1984 standardisiert
und ist ebenfalls ein �ffentliches, aber leitungsvermitteltes Netz.
Prinzipiell gibt es zwei Arten von Anschl�ssen:
\begin{itemize}
\item Der Basisanschluss (Basic Rate Interface, BRI) fasst zwei B Kan�le
und einen D-Kanal zusammen. Ein B-Kanal hat eine Datenrate von 64KBit/s
und ein D-Kanal eine Datenrate von 16KBit/s. B-Kan�le sind Datenkan�le
und der D-Kanal ist als Signalisierungskanal vorgesehen (Steuerung).
\item Der Prim�rmultiplexanschluss (Primary Rate Interface, PRI) fasst 30
B-Kan�le und einen D-Kanal zusammen.
\end{itemize}

\section{Frame Relay}

Bei Frame Relay handelt es sich um eine Weiterentwicklung von X.25.
Dazu kann es einerseits als abgemagerte Variante von X.25 betrachtet
werden, die h�here Datenraten unterst�tzt (bis zu 45MBit/s sind machbar).
Das ist deshalb m�glich, da X.25 f�r �bertragungsstrecken mit hohen
Fehlerraten ausgelegt war und deshalb aufw�ndige Protokolle f�r die
sichere �bertragung beinhaltet. Durch die wesentlich geringeren Fehlerraten
auf neueren digitalen �bertragungssystemen k�nnen einfachere Protokolle
verwendet werden. Charakteristisch ist, dass Pakete im Vergleich zu
ATM eine variable L�nge aufweisen.


\section{ATM}

ATM (asynchronous transfer mode) ist eine verbindungsorientierte,
die eine eine Bereitstellung von Dienstg�ten zur Verf�gung stellt.
ATM wurde 1986 standardisiert und wird haupts�chlich von Netzbetreibern
im WAN Bereich eingesetzt. Im LAN hat es sich nicht durchgesetzt und
mit der Einf�hrung von 10-Gigabit-Ethernet wird vermutet, dass ATM
auch im WAN (teilweise) abgel�st wird, da die Hardware relativ aufw�ndig
ist und auch eine umfassende Administration des Netzes notwendig ist.

Auch bei ATM k�nnen entweder permanente virtuelle Verbindungen (in
etwa wie Standleitungen) oder transiente virtuelle Verbindungen (in
etwa wie W�hlleitungen) aufgebaut werden. Eine Verbindung wird bei
ATM nur aufgebaut, wenn die geforderten G�teklassen erf�llt werden.
Bei ATM handelt es sich um eine zellenbasierte Technologie. Unter
einer Zelle versteht man ein kleines Paket mit fester Gr��e. Die Daten
werden dazu in Paket je 48 Bytes unterteilt und mit 5 Bytes Header
versehen. Der Vorteil solcher Pakete liegt darin, dass der Jitter
klein ist. ATM unterst�tzt derzeit Datenraten bis zu 625MBit/s.


\section{Zugangsnetze}

Als leitungsgebundene Zugangsnetze bzw. Zugangsverfahren sind in Verwendung:
Modemverbindung �ber Telefonleitung, ISDN Anschluss, xDSL (digital
subscriber line), Modem �ber Kabelnetz sowie die Mitverwendung der
Stromversorgungsleitungen.

F�r den leitungsungebundenen Zugang werden Mobilfunk �ber GSM, Mobilfunk
�ber UMTS, Richtfunk und Satellitenfunk eingesetzt. In der Zukunft
werden weitere Funktechnologien folgen, z.B.: Wimax (worldwide interoperability
for microwave access), ein Funkstandard f�r Breitbandanbindungen bis
zu 70MBit/s und bis zu 50km.
