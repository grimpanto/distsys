
\chapter{NAT\label{sec:NAT}}

NAT (Network Address Translation) ist in Rechnernetzen der Sammelbegriff
f�r Verfahren, um automatisiert und transparent Adressen in IP Paketen
durch andere Adressen zu ersetzen. Diese Adressumsetzung wird in der
Regel zwischen einem �ffentlichen und einem internen Netz durchgef�hrt.

Der Haupteinsatz von NAT besteht darin Netze mit privaten IP Adressen
an das �ffentliche Internet anzubinden und damit auch �ffentliche
IP Adressen zu sparen. In gewisser Weise kann es auch als eine Sicherheitsma�nahme
eingesetzt werden, auch wenn dies nicht die prim�re treibende Absicht
ist.

Das Prinzip ist, dass am Router, der das interne vom �ffentlichen
Netz trennt eine Umsetzungstabelle enthalten ist, die eine Abbildung
der privaten IP Adressen auf die �ffentlichen IP Adressen enth�lt.
Unter Umst�nden enth�lt diese Umsetzungstabelle weitere Informationen,
um die Abbildung eindeutig zu machen.

Um komplexere Protokolle (wie ftp oder VoIP) �ber einen NAT-Router
betreiben zu k�nnen, muss der NAT-Router allerdings als ein Gateway
fungieren.

NAT hat trotz des komplexeren Systemaufbaus sicher dazu beigetragen,
dass die Anzahl der freien IPv4 Adressen viel langsamer abgenommen
hat.

Es gibt zwei grundlegende Formen von NAT, die in den beiden folgenden
Abschnitten kurz besprochen werden.


\section{Source-NAT}

Beim Source-NAT (SNAT) wird bei einer Anfrage eines Clients im lokalen
Netz an einen Server im Internet die private Client-Adresse auf eine
verf�gbare externe Adresse ge�ndert. Es gibt prinzipiell drei M�glichkeiten:
\begin{labeling}{00.00.0000}
\item [{Statisches~SNAT}] Hierbei wird jeder privaten Adresse genau eine
externe Adresse zugeordnet. D.h. es gibt eine 1:1 Abbildung von lokalen
zu extern verf�gbaren Adressen. Dies kann notwendig sein, wenn der
interne Rechner als Server im Internet verf�gbar sein soll oder wenn
man die Sicherheit erh�hen will (Struktur des internen Netzes wird
nicht preisgegeben).
\item [{Dynamisches~SNAT}] Sind mehr lokale Rechner vorhanden als externe
Adressen zur Verf�gung stehen, muss die Zuordnung von lokaler zu externer
Adresse dynamisch vorgenommen werden. Die Zuordnungstabelle im NAT-Router
kann noch mit zus�tzlichen Informationen wie z.B. Portangaben versehen
werden, sodass auch gleichzeitig mehr interne Rechner mit Servern
im Internet kommunizieren k�nnen als externe Adressen zur Verf�gung
stehen.
\item [{Masquerading}] Es handelt sich um eine Sonderform bei der alle
internen Adressen auf \emph{eine} externe Adresse abgebildet werden.
\end{labeling}

\section{Destination-NAT}

Das Destination-NAT (DNAT) ist im Grunde die Umkehrung des SNAT: Es
wird die Ziel-Adresse abgebildet. Ein externer Client sendet eine
Anfrage an eine extern verf�gbare Adresse unseres Netzes (z.B. an
die Adresse des NAT-Routers). Der NAT-Router bildet die Zieladresse
z.B. auf die interne Adresse des DMZ Servers ab und leitet daher diese
Anfrage an den DMZ Server weiter. Hier gibt es prinzipiell zwei M�glichkeiten:
\begin{itemize}
\item Der NAT-Router leitet alle Anfragen bzgl. einer extern verf�gbaren
Adresse an den DMZ Server weiter.
\item Der NAT-Router leitet Anfragen bzgl. einer extern verf�gbaren Adresse
an die externe Adresse weiter, wenn zus�tzliche Kriterien zutreffen.
Das wichtigste Kriterium ist die Portangabe. D.h. nur wenn die Portangabe
z.B. 80 ist wird an den internen DMZ Server weitergeleitet.
\end{itemize}

