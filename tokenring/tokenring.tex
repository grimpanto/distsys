
\chapter{Tokenring}

Die Tokenring Technologie wurde urspr�nglich von IBM entwickelt und
ist in IEEE 802.5 spezfiziert. Urspr�nglich waren entweder 4MBit/s
(Ringl�nge max. 360m) bzw. 16MBit/s (Ringl�nge max. 168m) Datenrate
m�glich. Bez�glich der �bertragungsleistung entspricht dies in etwa
der �bertragungsleistung eines Ethernet-Netzes mit 10MBit/s bzw. 100MBit/s.
Aktuelle Weiterentwicklungen sind z.B. die GBit-Variante (IEEE 802.5v),
ein Konzept zur Realisierung von VLANs bzw. das Konzept der Link Aggregation.
Link Aggregation bedeutet die Verbindung zweier Switches durch parallele
Links zur Erh�hung der Bandbreite.


\section{Prinzip}

Das Konzept besteht darin, dass ein logischer Ring der Arbeitsstationen
gebildet wird. In diesem Ring wird ein Token (dt. Marke) im Kreis
geschickt. Solange keine Station senden will, kreist ein Frei-Token.
Will eine Station senden, nimmt es das Token aus dem Ring und ersetzt
es durch ein Besetzt-Token und h�ngt die Daten und die Adresse an
und schickt das Token wieder in den Ring. Kommt das Token beim Empf�nger
an, wird das Token von diesem kopiert und wieder in den Ring eingespeist.
Erst der Sender entnimmt das Token aus dem Ring und ersetzt es wieder
durch ein Frei-Token. Jeder Sender darf maximal f�r die Dauer der
THT (token holding time) senden (typisch 10ms), danach muss er ein
Frei-Token weitergeben.

Es gibt also keine ausgezeichnete Station, die den Zugriff steuert.
D.h. der Tokenring verwendet ein dezentrales Zuteilungsverfahren.
Zur �berwachung des Ringes wird eine Station als aktiver Monitor bestimmt,
w�hrend andere Stationen als passive Monitore (standby monitors) agieren
k�nnen, die bei Ausfall des aktiven Monitors selbst aktiv werden.


\section{Netzaufbau}

Die Stationen werden aktiv in den Ring eingef�gt. Sie empfangen und
regenerieren das am Eingang ankommende Signal, werten es aus und geben
es regeneriert am Ausgang weiter. F�llt eine Station aus oder wird
sie ausgeschaltet, w�rde allerdings der Ring unterbrochen werden.
Um dies zu verhindern, werden die Stationen \emph{sternartig} an einen
Ringleitungsverteiler, genannt auch MSAU (multi station access unit)
in den Ring eingef�gt.

In jeder MSAU sind Schalter eingebaut, die bei Ausfall oder Abschaltung
einer Station diese �berbr�cken und dadurch den Ring wieder herstellen.
D.h. z.B., dass die Verkabelung sternartig durchgef�hrt wird, aber
das Netz logisch als Ring arbeitet.
