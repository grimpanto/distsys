
\chapter{IPv6}

Die wesentlichen funktionellen Unterschiede zu IPv4 sind:
\begin{itemize}
\item Adressraum: Einer der Beweggr�nde IPv6 zu entwickeln, war die geringe
Anzahl an freien Adressbereichen. IPv6 hat 128 Bit lange Adressen.
D.h. es gibt ca. $3.4\cdot10^{38}$ m�gliche Adressen. Die Notation
wird nicht mehr byteweise in Dezimalschreibweise durch Punkte getrennt,
sondern wortweise (16 Bit) in Hexadezimalschreibweise durch Doppelpunkte
getrennt angegeben. Z.B.: 2002:89d0:e02e::89d0:e02e.
\item Durch die hohe Anzahl an verschiedenen Adressen besteht die M�glichkeit
diese geographisch-hierarchisch zu vergeben. Dadurch k�nnen die Anzahl
der Routereintr�ge weiter gesendet werden.
\item Es wurde eine lokale Adressvergabe eingef�hrt, sodass Adressen automatisch
vergeben werden. Dazu kann sich ein Host eine IPv6 Adresse direkt
aus einer MAC Adresse geben und mit angrenzenden Hosts und Routern
diesbez�glich abstimmen.
\item Es gibt im IP Header keine Checksumme mehr. D.h. alle �berpr�fungen
m�ssen durch h�here Protokolle erledigt werden. Das liegt daran, dass
viele Router die Checksum �berhaupt nicht �berpr�ft haben, sondern
einfach um eins erh�ht (wg. dem Header TTL) haben.


Der Basisheader wurde generell k�rzer gestaltet. Er enth�lt nur mehr
7 anstatt 13 Felder. Es kann allerdings auch mehrere Header geben.
Diese �nderungen erm�glichen Routern Paket schneller zu verarbeiten.

\item Sicherheitsfunktionen wurden direkt eingebaut und basieren auf IPSec.
IPSec kann zwar auch mit IPv4 verwendet werden, ist aber in IPv6 direkt
integriert.
\item Die Fragmentierung der Datenpakete in Routern wurde entfernt. M�sste
ein Paket fragmentiert werden, dann wird eine Fehlermeldung an den
Sender zur�ckgeschickt, der daraufhin die MSS anpassen muss und erneut
sendet.
\item QoS wurde ebenfalls integriert. Ziel ist die verbesserte �bertragung
von Audio und Videodaten sowie die �bertragung von Daten in Echtzeit.
\item Weiters gibt es zus�tzlich zu Broadcast und Multicast auch Anycast-Adressen.
\end{itemize}

